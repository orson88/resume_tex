%-------------------------
% Resume in Latex
% Author : Sourabh Bajaj + some brand new features from Mary Feofanova + some changes by Muhammadjon Hakimov
% Sourabh's: https://github.com/sb2nov/resume
% Mary's: https://github.com/mary3000/resume
% Muhammadjon's: https://github.com/mrhakimov/resume
% License : MIT
%------------------------

\documentclass[letterpaper,10pt]{article}

\usepackage{makecell}
\usepackage[link=off]{phonenumbers}
\usepackage{ragged2e}

\usepackage{latexsym}
\usepackage[empty]{fullpage}
\usepackage{titlesec}
\usepackage{marvosym}
\usepackage[usenames,dvipsnames]{color}
\usepackage{verbatim}
\usepackage{enumitem}
\usepackage[pdftex]{hyperref}
\usepackage{fancyhdr}
\usepackage[T2A,T1]{fontenc}
\usepackage[utf8]{inputenc}
\usepackage[russian]{babel}

\pagestyle{fancy}
\fancyhf{} % clear all header and footer fields
\fancyfoot{}
\renewcommand{\headrulewidth}{0pt}
\renewcommand{\footrulewidth}{0pt}
\usepackage[margin=0.3in]{geometry}
% Adjust margins
\addtolength{\oddsidemargin}{-0.0in}
\addtolength{\evensidemargin}{-0.0in}
\addtolength{\textwidth}{0in}
\addtolength{\topmargin}{10pt}
\addtolength{\textheight}{0.0in}

\urlstyle{same}

\usepackage{xcolor} % http://ctan.org/pkg/xcolor
\usepackage{hyperref} % http://ctan.org/pkg/hyperref
\hypersetup{
  colorlinks=true,
  linkcolor=blue!50!red,
  linkbordercolor=red,
  urlcolor=black!70!black,
  pdfnewwindow=true
}

\raggedbottom
\raggedright
\setlength{\tabcolsep}{0in}

% Sections formatting
\titleformat{\section}{
  \vspace{-10pt}\scshape\raggedright\large
}{}{0em}{}[\color{black}\titlerule \vspace{-7pt}]

%-------------------------
% Custom commands
\def \ifempty#1{\def\temp{#1} \ifx\temp\empty }

\newcommand{\resumeItem}[2]{
  \item\small{
  	\ifempty{#1}#2\else\textbf{#1}{: #2 \vspace{-2pt}}\fi
  }
}

\usepackage[dvipsnames]{xcolor}

\usepackage{lmodern}
\usepackage{tikz}

% Style definition
\tikzset{rndblock/.style={rounded corners,rectangle,draw,outer sep=0pt}}

% Command Definition
% 1 optional to customize the aspect, 2 mandatory: text to be framed
\newcommand{\tframed}[2][]{\tikz[baseline=(h.base)]\node[rndblock,#1] (h) {\color{black}{#2}};}

\newcommand*{\mystrut}{\rule[-0.2\baselineskip]{0pt}{0.8\baselineskip}}
\newcommand{\skill}[1]{\tframed[lightgray]{\mystrut#1}}


\newcommand{\resumeSubheading}[4]{
  \item
    \begin{tabular*}{0.97\textwidth}{l@{\extracolsep{\fill}}r}
      \vspace{-12pt}\textbf{#3} & \textcolor{mygray}{\textit{\small #2}} \\
      \textit{\small#1} & \textcolor{mygray}{\textit{\small #4}} \\
    \end{tabular*}\vspace{-5pt}
}

\newcommand{\resumeExpSubheading}[5]{
  \vspace{3pt}
  \item
    \begin{tabular*}{0.97\textwidth}{l@{\extracolsep{\fill}}r}
      \vspace{2pt} \textbf{#1}  & \textcolor{mygray}{\small #2} \\
      \textit{#3} & \textcolor{mygray}{\textit{\small #4}} \\
      {\scriptsize#5}
    \end{tabular*}\vspace{3pt}
}

\newcommand{\resumeExpJointSubheading}[5]{
  \vspace{-14pt}
  \item
    \begin{tabular*}{0.97\textwidth}{l@{\extracolsep{\fill}}r}
      \vspace{2pt} \textbf{#1}  & \textcolor{mygray}{\small #2} \\
      \textit{#3} & \textcolor{mygray}{\textit{\small #4}} \\
      {\scriptsize#5}
    \end{tabular*}\vspace{3pt}
}

\newcommand{\resumeProjSubheading}[5]{
  \vspace{-10pt}\item
    \begin{tabular*}{0.97\textwidth}{l@{\extracolsep{\fill}}r}
      \vspace{2pt} \textbf{#1}  & \textcolor{mygray}{\small #2} \\
      \textbf{#3} & \textcolor{mygray}{\textit{\small #4}} \\
      {\scriptsize#5}
    \end{tabular*}\vspace{3pt}
}

\newcommand{\resumeSubItem}[2]{\resumeItem{#1}{#2}\vspace{-4pt}}

\renewcommand{\labelitemii}{$\circ$}

\newcommand{\resumeSubHeadingListStart}{\begin{itemize}[leftmargin=*]}
\newcommand{\resumeSubHeadingListEnd}{\end{itemize}}
\newcommand{\resumeItemListStart}{\begin{itemize}[leftmargin=0.2in]}
\newcommand{\resumeItemListEnd}{\end{itemize}\vspace{-5pt}}

\usepackage{changepage}
\newcommand{\resumeDesc}[1]{\begin{adjustwidth}{5pt}{0pt}\vspace{-2pt}{#1}\end{adjustwidth}}

\newcommand{\ExternalLink}{
    \tikz[x=1.2ex, y=1.2ex, baseline=-0.05ex]{
        \begin{scope}[x=1ex, y=1ex]
            \clip (-0.1,-0.1) 
                --++ (-0, 1.2) 
                --++ (0.6, 0) 
                --++ (0, -0.6) 
                --++ (0.6, 0) 
                --++ (0, -1);
            \path[draw, 
                line width = 0.5, 
                rounded corners=0.5] 
                (0,0) rectangle (1,1);
        \end{scope}
        \path[draw, line width = 0.5] (0.5, 0.5) 
            -- (1, 1);
        \path[draw, line width = 0.5] (0.6, 1) 
            -- (1, 1) -- (1, 0.6);
        }
    }
    
\definecolor{Blue1}{HTML}{4D4EDC}
\newcommand{\MYhref}[3][Blue1]{\href{#2}{\color{#1}{#3}}}

%-------------------------------------------
%%%%%%  CV STARTS HERE  %%%%%%%%%%%%%%%%%%%%%%%%%%%%


\begin{document}
%----------HEADING-----------------

\begin{center}\textbf{\Large Поздняков Арсений}\end{center}
\vspace{-12pt}
\begin{center}
Email: \MYhref{mailto:malecsenya@gmail.com}{malecsenya@gmail.com} \quad
Telegram: \MYhref{https://www.t.me/tkrfrf}{@tkrfrf} \quad
GitHub: \MYhref{https://www.github.com/orson88}{orson88}
\end{center}

%-----------SUMMARY-----------------
\vspace{-10pt}
\section{Обо мне}
\resumeSubHeadingListStart
\justifying
Data Scientist с 1+ годом опыта в аналитике и инженерии данных. Хотел бы работать Data Scietist-ом или Machine Learning Инженером в команде по рекомендательным или скоринговым системам. Люблю все что связано в DS и ML с промышленной разработкой, а также заниматься полным циклом ML/DS решений: от проектирования до внедрения в продакшн.
\resumeSubHeadingListEnd

%-----------EXPERIENCE-----------------
\vspace{-5pt}
\section{Опыт}
\justifying
  \resumeSubHeadingListStart
   \resumeExpSubheading
      {\href{https://www.mntk.ru}{МНТК им. С. Фёдорова -- офтальмологическая клиника/исследовательский институт  \ExternalLink}}{Москва, Россия}
      {Data Engineer (проект)}{Ноябрь 2022 --- Январь 2023}
      {\skill{PostgreSQL} \skill{Python} \skill{AirFlow} \skill{PowerBI} \skill{BI Visiology} \skill{DWH} \skill{Docker}}
      \resumeDesc{
      \begin{itemize}
          \item Спроектировал \textbf{макеты лаконичных дашбордов} используя \underline{Power BI} для дальнейшей отрисовки в \underline{BI Visiology}. 
          \item С помощью \underline{PostgreSQL, Python \& Airflow} \textbf{с нуля настроил хранилище данных}, автоматизированное и оркестрируемое, конечной целью которого являются аналитические витрины под дашборды и отчетности.
          \item Поставил на конвеер всю систему и запустил готовую BI-систему с дашбордами из \underline{Docker-контейнера} для внутреннего пользования
      \end{itemize}}
      
    \resumeExpSubheading
      {\href{https://https://ac.gov.ru}{Аналитический Центр при Правительстве РФ --- некоммерческая аналитическая организация\ExternalLink}}{Москва, Россия}
      {Junior Data Scientist}{Ноябрь 2021 --- Август 2022}
      {\skill{Машинное Обучение} \skill{Python} \skill{Временные ряды} \skill{SQL} \skill{nifi} \skill{Обучение стажеров}}
      \resumeDesc{
      \begin{itemize}
          \item Добавил на дашборды почти всех продуктов полученные с помощью \underline{эконометрических и регрессионных моделей} прогнозные знанчения, позволившие аналитикам и руководству принимать дальнейшие решения на административном уровне.
          \item Построил \underline{интепретируемую} регрессионную модель, которая спасла много человеческих жизней, \break
          \underline{получил благодарственное письмо} от члена Правительства РФ.
          \item Автоматизировал большую часть отчетностей с помощью \underline{Python}-пайплайнов и обучил несколько стажеров минимальному уровню python для аналитики и отчетностей. (pandas, numpy)
      \end{itemize}}

  \resumeSubHeadingListEnd

%--------PROGRAMMING SKILLS------------
\section{Навыки}
 \resumeSubHeadingListStart
 \begin{tabular}{ll}
\textbf{Языки:} & \quad Python, SQL, GO (базовый) \\
\textbf{Библиотеки \& Фреймворки:} & \quad pandas, sklearn, plotly, spark, mlflowoptuna, xgboost, catboost, shap, NetworkX, \break \\
\textbf{} & \quad   базовые TensorFlow и PyTorch, psycopg2 \& etc., FastAPI \\
\textbf{Инструменты:} & \quad Jira, Docker, PyCharm, DataSpell, DataGrip, Linux, Bash, Git \\
\end{tabular}
 \resumeSubHeadingListEnd

%-----------EDUCATION-----------------
\section{Образование}
 \resumeSubHeadingListStart
 \begin{tabular}{ll}
 \textbf{Магистратура, 2022-2024:} & \quad Машинное обучение и высоконагруженные системы, \break \\
\textbf{} & \quad Факультет Компьютерных Наук, \break \\
\textbf{} & \quad Высшая Школа Экономики \\
\textbf{} & \quad \\
\textbf{Бакалавриат, 2018-2022:} & \quad Прикладная математика и информатика, \break \\
\textbf{} & \quad Факультет Информационных технологий и анализа больших данных, \break \\
\textbf{} & \quad Финансовый Университет при Правительстве РФ \\


\end{tabular}
 \resumeSubHeadingListEnd

  
%-----------PROJECTS-----------------
\section{Проекты}
  \resumeSubHeadingListStart
      \resumeProjSubheading
      {}{}{{Пет-проекты, исследовательские проекты и курсовые работы}}{}

          \resumeDesc{
          \begin{itemize}
              \item Веб-приложение для отслеживания объявлений машин на авто.ру, написанное с помощью streamlit [\href{https://github.com/orson88/micro_autoru_app}{github}]
              \item Emotion Recognition голосовая модель, созданная в кооперации со студентами из американского института \break
              часть этого проекта можно увидеть тут [\href{https://github.com/orson88/TeleMedicineShowcase}{github}]
              \item Проверка пренадлежности лог-доходности акций нормальному распределению тестом Харке-Бера
              \item Классификация fake/true новостей в твиттере с помощью GLOVE эмбеддингов
              \item Прогнозирование оттока клиентов телекоммуникационной компании
              \item Долгосрочная алгоритмическая торговая стратегия на основании классификационной модели и дивидендных гэпов
          \end{itemize}
          }
          

  \resumeSubHeadingListEnd


\end{document}
